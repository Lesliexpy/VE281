\documentclass[a4paper]{article}

\usepackage[T1]{fontenc}
\usepackage[margin=0in]{geometry}
\usepackage{tikz}
\usepackage{listings}

\lstset{
language=C,
basicstyle=\small\sffamily,
numbers=left,
numberstyle=\tiny,
frame=tb,
columns=fullflexible,
showstringspaces=false
}


\begin{document}
\pagestyle{empty}
  \noindent   \begin{tikzpicture}[x=1cm,y=1cm]
\foreach \x in {0,10.5,21}
  {
    \draw (\x ,0) -- (\x ,29.65);
  };
\foreach \y in {0,7.41,14.83,22.24,29.65}
  {
    \draw (0, \y) -- (21, \y);
  };

 \foreach \x in {0,10.5}
  {
 \foreach \y in {0,7.41,14.83,22.24}
  {  
\draw(0.5+\x,3.75+\y) node [right,text width=9.5cm] {
{\bf  BTS SIO 1\`ere ann\'ee}\ \newline
{\bf  Correction pour le tri de trois nombres}\ \ 

\begin{lstlisting}
#include <iostream>

using namespace std;

int main(int argc, const char * argv[])
{

    // insert code here...
    int a,b,c,ancien;
    cout << "Entrez le premier nombre\n";
    cin >> a;
    cout << "Entrez le second nombre\n";
    cin >> b;
    cout << "Entrez le troisième nombre\n";
    cin >> c;
    if (b<a) {
        ancien=a;
        a=b;
        b=ancien;
    }
    if (c<b) {
        ancien=b;
        b=c;
        c=ancien;
    }
    if (b<a) {
        ancien=a;
        a=b;
        b=ancien;
    }
    cout << "Voici les nombres dans l'ordre croissant : \n" << a << "," << b << "," << c << "\n";
    return 0;
}
\end{lstlisting}


};
};

\end{tikzpicture}
\end{document}